\documentclass[12pt]{letter}
\usepackage{stdhw,multicol,uncial}
\usepackage[colorlinks=true,urlcolor=black]{hyperref}

%\usepackage{humanist}
%\usepackage[T1]{fontenc}

\pagestyle{empty}
\renewcommand{\normalfont}{\sffamily}

\geometry{hmargin=1in, vmargin=0.8in}

\renewcommand{\statebold}[1]{\textbf{\boldmath{#1}}:\ }
\renewcommand{\labelitemi}{--}

\newcommand{\heading}[1]{\hspace{-0.3in}\textbf{#1}\:\,$\triangleright$\quad}  % add a right arrow here

\newcommand{\importantnote}[1]{\vspace{-0.5em}\begin{itemize} \item[$\rightarrow$]
	\textit{#1}
\end{itemize}\vspace{-0.5em}}

\begin{document}
\normalfont
\vskip-12pt
\begin{center}
{\large\unclfamily Math 307B $\cdot$ Spring 2013}\\
MWF 9:30--10:20, in EEB 037\\
\url{http://www.math.washington.edu/~pcaday/307/}
\end{center}

%%%% hanging indent: from http://www.wkiri.com/today/?p=76
%\leftskip 0.3in
%\parindent -0.3in
%%%

\leftskip 0.3in

\heading{Contact Info} \parbox[t]{5in}{Peter Caday (\texttt{pcaday@uw.edu})\\Office: Padelford C-34}

\heading{Class Website} The class website will contain important announcements, study material, and homework assignments. It's important for you to check it regularly. I will also send information by email to the class.

\heading{Textbook} \textit{Elementary Differential Equations and Boundary Value Problems}, 9th edition, by William E.~Boyce.

\heading{Office Hours} Office hours are your chance to get one-on-one help with homework and ask questions about the homework. We'll decide on office hours as a class, and there is an survey online (linked from the course website) where you can indicate what times work best for you.

This quarter, several of the 307 instructors will be sharing office hours, meaning that you can come to any of our office hours to get help on homework or ask questions on the class. We'll post a schedule online with all the office hours you can attend.

Office hours will be held in the conference room in the Math Study Center, in the basement of Communications, room B-014. When you enter the main room, look to the right of the front desk and you'll find the conference room.

%\importantnote{Since we don't have quiz sections in 307, office hours are your time to get one-on-one help with the material.}

\heading{Help with 307} Whether you have questions on homework or lecture, or find yourself struggling with the class, don't be afraid to ask for help. You can email me, come by office hours, or make an appointment.

Another place you can get free, one-on-one tutoring is CLUE. They hold regular drop-in tutoring sessions; for more information, see their website at
\begin{center}
\url{http://depts.washington.edu/clue}.
\end{center}

\heading{Homework} We will have around 7 homework assignments this quarter, which will be posted on the class website, under the ``Homework'' section. You may work on the homework with other Math 307 students, but you must write up your solutions separately. Homework assignments will normally be turned in during class.

\importantnote{Always show your work! Without proper work shown, you may not receive credit, even if you have the correct answer.}

Homework will usually be due in class. If you are unable to attend class, you can have a friend turn your homework in, or you can scan and email it to me \emph{before class}. You are allowed to turn in one homework assignment up to one week late (except for the last homework), although 10\% will be deducted from that homework's score. After the first late homework, I will look over your work, but you won't receive any credit for it.

To help our grader, make sure to \textbf{staple} your homework, and clearly label your problems.

\pagebreak

\heading{Quizzes} There will be 5--7 short quizzes throughout the quarter, usually on Fridays. Quizzes are meant to check whether you have a basic understanding of the material from that week. Topics for each quiz will be announced a few classes ahead of time. 

\heading{Exams}
We will have two midterms and a final, in our regular classroom. The midterms are scheduled tentatively for Monday, April 22, and Monday, May 20, during our regular class times.

Our final will be cumulative, but emphasizing material not covered by the first two midterms, and is scheduled for Wednesday, June 12, 8:30--10:20.

\heading{Grade Breakdown}
\begin{tabular}[t]{ll}
	Quizzes & 10\%\\
	Homework \qquad\qquad & 15\%\\
	Midterm 1 & 20\%\\
	Midterm 2 & 20\%\\
	Final & 35\%
\end{tabular}
%\parbox[t]{5in}{}

\heading{Other Information}
If you have special circumstances, such as a disability, which require accommodation, let me know, and I would be glad to help.

\end{document}
